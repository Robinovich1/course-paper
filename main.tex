\documentclass[14pt]{article}
% \usepackage[14pt]{extsizes}
\usepackage[utf8]{inputenc}
\usepackage{amssymb,amsthm}
\usepackage{amsmath}
\theoremstyle{definition}
\newtheorem{definition}{Определение}
\usepackage[russian]{babel}
\usepackage[shortlabels]{enumitem}
\newtheorem{theorem}{\bf Теорема}
\usepackage{graphicx}
\usepackage[left=3cm, right=3cm, top=3cm, bottom=3cm]{geometry}
\setlength{\parindent}{0cm}
\newenvironment{ourproof}{\\ \textit{Доказательство.}\\ }{$\hfill \heartsuit$}
\usepackage{ amssymb }
\usepackage{ dsfont }
\newtheorem{exercise}{Упражнение}
\newtheorem{example}{Пример}
\setlength{\parindent}{5ex}
\newtheorem{rem}{Замечание}[section]
\newtheorem{proposition}{Предложение}[section]
\usepackage[T1]{fontenc}
\usepackage{ mathrsfs }
\usepackage{mathrsfs}
\usepackage{ upgreek }
\usepackage{wrapfig}
\usepackage{textcomp}

\linespread{1.5} 
\frenchspacing

    

\usepackage{xcolor}
\usepackage{hyperref}


\begin{document}


\begin{titlepage}
  \begin{center}
    \normalsize
   \textbf {Правительство Российской Федерации\\ 
Федеральное государственное автономное образовательное учреждение\\
   высшего профессионального образования\\
    «Национальный исследовательский университет» \\
     «Высшая школа экономики»}
   

    
    
    
    \textbf {Нижегородский филиал}
    
  \vfill
    Факультет математики, информатики и компьютерных наук\\
    Кафедра компьютерных наук
    
   
    \vfill

    \textbf{ КУРСОВАЯ РАБОТА}\\[5mm]
    
    {\normalsize  \textbf{О кодовых LLM и способах оценки качества моделей для класса задач
генерации кода}}
    
  \bigskip
    
    
\end{center}
\vfill

\newlength{\ML}
\settowidth{\ML}{«\underline{\hspace{0.7cm}}» 
\underline{\hspace{2cm}}}
\hfill
\begin{minipage}{0.5\textwidth}
  Выполнил:\\
 Студент 2 курса группы 23КНТ7    
   \\
 Косульников Дмитрий Алексеевич\\ 


 \\Научный руководитель:\\
 Доцент кафедры\\ 
 ыфвфывывыв\\
ыфвфывыв\\
  \vspace{1cm}
 {\hspace{2.5cm}}
\end{minipage}%
\vfill

\begin{center}
  Нижний Новгород\\Май 2025 г.
\end{center}

\end{titlepage}

\pagebreake[2]


\newpage
\tableofcontents
\newpage
\section{Введение}

 
 
 \section{part 1}
 


\newpage
 \section{part 2}
 
 
 




\newpage
\section{Заключение}





\newpage
\addcontentsline{toc}{section}{\bf{Список литературы}}
\begin{thebibliography}{99}




\end{thebibliography}

\end{document}